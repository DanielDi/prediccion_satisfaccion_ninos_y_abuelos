% Options for packages loaded elsewhere
\PassOptionsToPackage{unicode}{hyperref}
\PassOptionsToPackage{hyphens}{url}
%
\documentclass[
]{article}
\usepackage{amsmath,amssymb}
\usepackage{lmodern}
\usepackage{iftex}
\ifPDFTeX
  \usepackage[T1]{fontenc}
  \usepackage[utf8]{inputenc}
  \usepackage{textcomp} % provide euro and other symbols
\else % if luatex or xetex
  \usepackage{unicode-math}
  \defaultfontfeatures{Scale=MatchLowercase}
  \defaultfontfeatures[\rmfamily]{Ligatures=TeX,Scale=1}
\fi
% Use upquote if available, for straight quotes in verbatim environments
\IfFileExists{upquote.sty}{\usepackage{upquote}}{}
\IfFileExists{microtype.sty}{% use microtype if available
  \usepackage[]{microtype}
  \UseMicrotypeSet[protrusion]{basicmath} % disable protrusion for tt fonts
}{}
\makeatletter
\@ifundefined{KOMAClassName}{% if non-KOMA class
  \IfFileExists{parskip.sty}{%
    \usepackage{parskip}
  }{% else
    \setlength{\parindent}{0pt}
    \setlength{\parskip}{6pt plus 2pt minus 1pt}}
}{% if KOMA class
  \KOMAoptions{parskip=half}}
\makeatother
\usepackage{xcolor}
\IfFileExists{xurl.sty}{\usepackage{xurl}}{} % add URL line breaks if available
\IfFileExists{bookmark.sty}{\usepackage{bookmark}}{\usepackage{hyperref}}
\hypersetup{
  pdflang={es},
  hidelinks,
  pdfcreator={LaTeX via pandoc}}
\urlstyle{same} % disable monospaced font for URLs
\usepackage[margin=1in]{geometry}
\usepackage{color}
\usepackage{fancyvrb}
\newcommand{\VerbBar}{|}
\newcommand{\VERB}{\Verb[commandchars=\\\{\}]}
\DefineVerbatimEnvironment{Highlighting}{Verbatim}{commandchars=\\\{\}}
% Add ',fontsize=\small' for more characters per line
\usepackage{framed}
\definecolor{shadecolor}{RGB}{248,248,248}
\newenvironment{Shaded}{\begin{snugshade}}{\end{snugshade}}
\newcommand{\AlertTok}[1]{\textcolor[rgb]{0.94,0.16,0.16}{#1}}
\newcommand{\AnnotationTok}[1]{\textcolor[rgb]{0.56,0.35,0.01}{\textbf{\textit{#1}}}}
\newcommand{\AttributeTok}[1]{\textcolor[rgb]{0.77,0.63,0.00}{#1}}
\newcommand{\BaseNTok}[1]{\textcolor[rgb]{0.00,0.00,0.81}{#1}}
\newcommand{\BuiltInTok}[1]{#1}
\newcommand{\CharTok}[1]{\textcolor[rgb]{0.31,0.60,0.02}{#1}}
\newcommand{\CommentTok}[1]{\textcolor[rgb]{0.56,0.35,0.01}{\textit{#1}}}
\newcommand{\CommentVarTok}[1]{\textcolor[rgb]{0.56,0.35,0.01}{\textbf{\textit{#1}}}}
\newcommand{\ConstantTok}[1]{\textcolor[rgb]{0.00,0.00,0.00}{#1}}
\newcommand{\ControlFlowTok}[1]{\textcolor[rgb]{0.13,0.29,0.53}{\textbf{#1}}}
\newcommand{\DataTypeTok}[1]{\textcolor[rgb]{0.13,0.29,0.53}{#1}}
\newcommand{\DecValTok}[1]{\textcolor[rgb]{0.00,0.00,0.81}{#1}}
\newcommand{\DocumentationTok}[1]{\textcolor[rgb]{0.56,0.35,0.01}{\textbf{\textit{#1}}}}
\newcommand{\ErrorTok}[1]{\textcolor[rgb]{0.64,0.00,0.00}{\textbf{#1}}}
\newcommand{\ExtensionTok}[1]{#1}
\newcommand{\FloatTok}[1]{\textcolor[rgb]{0.00,0.00,0.81}{#1}}
\newcommand{\FunctionTok}[1]{\textcolor[rgb]{0.00,0.00,0.00}{#1}}
\newcommand{\ImportTok}[1]{#1}
\newcommand{\InformationTok}[1]{\textcolor[rgb]{0.56,0.35,0.01}{\textbf{\textit{#1}}}}
\newcommand{\KeywordTok}[1]{\textcolor[rgb]{0.13,0.29,0.53}{\textbf{#1}}}
\newcommand{\NormalTok}[1]{#1}
\newcommand{\OperatorTok}[1]{\textcolor[rgb]{0.81,0.36,0.00}{\textbf{#1}}}
\newcommand{\OtherTok}[1]{\textcolor[rgb]{0.56,0.35,0.01}{#1}}
\newcommand{\PreprocessorTok}[1]{\textcolor[rgb]{0.56,0.35,0.01}{\textit{#1}}}
\newcommand{\RegionMarkerTok}[1]{#1}
\newcommand{\SpecialCharTok}[1]{\textcolor[rgb]{0.00,0.00,0.00}{#1}}
\newcommand{\SpecialStringTok}[1]{\textcolor[rgb]{0.31,0.60,0.02}{#1}}
\newcommand{\StringTok}[1]{\textcolor[rgb]{0.31,0.60,0.02}{#1}}
\newcommand{\VariableTok}[1]{\textcolor[rgb]{0.00,0.00,0.00}{#1}}
\newcommand{\VerbatimStringTok}[1]{\textcolor[rgb]{0.31,0.60,0.02}{#1}}
\newcommand{\WarningTok}[1]{\textcolor[rgb]{0.56,0.35,0.01}{\textbf{\textit{#1}}}}
\usepackage{graphicx}
\makeatletter
\def\maxwidth{\ifdim\Gin@nat@width>\linewidth\linewidth\else\Gin@nat@width\fi}
\def\maxheight{\ifdim\Gin@nat@height>\textheight\textheight\else\Gin@nat@height\fi}
\makeatother
% Scale images if necessary, so that they will not overflow the page
% margins by default, and it is still possible to overwrite the defaults
% using explicit options in \includegraphics[width, height, ...]{}
\setkeys{Gin}{width=\maxwidth,height=\maxheight,keepaspectratio}
% Set default figure placement to htbp
\makeatletter
\def\fps@figure{htbp}
\makeatother
\setlength{\emergencystretch}{3em} % prevent overfull lines
\providecommand{\tightlist}{%
  \setlength{\itemsep}{0pt}\setlength{\parskip}{0pt}}
\setcounter{secnumdepth}{-\maxdimen} % remove section numbering
\ifLuaTeX
\usepackage[bidi=basic]{babel}
\else
\usepackage[bidi=default]{babel}
\fi
\babelprovide[main,import]{spanish}
% get rid of language-specific shorthands (see #6817):
\let\LanguageShortHands\languageshorthands
\def\languageshorthands#1{}
\usepackage{booktabs}
\usepackage{longtable}
\usepackage{array}
\usepackage{multirow}
\usepackage{wrapfig}
\usepackage{float}
\usepackage{colortbl}
\usepackage{pdflscape}
\usepackage{tabu}
\usepackage{threeparttable}
\usepackage{threeparttablex}
\usepackage[normalem]{ulem}
\usepackage{makecell}
\usepackage{xcolor}
\ifLuaTeX
  \usepackage{selnolig}  % disable illegal ligatures
\fi

\author{}
\date{\vspace{-2.5em}}

\begin{document}

\begin{titlepage}
\centering
{\scshape\Huge Informe Técnico \par}
{\scshape\Huge  Satisfacción de Niños y Abuelos \par}
\vspace{2cm}
{\Large Por: \par}
{\Large Daniel Espinal Mosquera\par}
{\Large Juan Sebastián Falcón\par}
{\Large Juan F. Peña Tamayo\par}
{\Large Brayan M. Ortiz Fajardo\par}
{\Large Thalea Marina Hesse\par}
\vfill
\vspace{0.5cm}


\begin{center}\includegraphics[width=300px,height=150px]{Imagenes/LOGO} \end{center}
\vfill
\vspace{0.5cm}
{\bfseries\LARGE Universidad Nacional de Colombia \par Sede Medellín \par}
\vspace{1cm}
\end{titlepage}

\hypertarget{introducciuxf3n}{%
\section{Introducción}\label{introducciuxf3n}}

La satisfacción de la vida y la felicidad varía entre países \cite{b7} y
juegan un papel importante en el desarrollo de un país. Sin embargo, no
se logró determinar documentación o estudios relacionados sobre cómo se
pueden predecir estos factores en Colombia dados unos parámetros. Por
tal motivo, se propone el plantemiento, desarrollo, análisis y posterior
productización de un modelo con el cual se busca predecir la
satisfacción de niños y abuelos. Para lograr esto, se toma como base una
encuesta realizada por el DANE en el 2020: Colombia - Encuesta Nacional
de Calidad de Vida - ECV 2020. Esta investigación, según el DANE ``Busca
cuantificar y caracterizar las condiciones socioeconómicas de los
hogares colombianos, con el fin de obtener la información necesaria para
la actualización de indicadores sociales a nivel de viviendas, hogares y
personas, y para la definición de políticas que permitan diseñar y
ejecutar planes sociales.'' (Metodologia ECV, 2009, p.17)\\
La estructura del estudio se planteó de la siguiente manera: se hizo una
búsqueda exhaustiva sobre documentación para determinar cuáles de las
variables que se tienen afectan de manera significativa la satisfacción.
Después se planteó un modelo general, sin embargo al revisar las
correlaciones entre las variables predictoras se dicidió partir ese
modelo general en tres sub-modelos: satisfacción de salud, seguridad y
trabajo. Para cada cada uno de estos también se realizó la búsqueda de
documentación al respecto. Adicionalmente, se creó una página web para
poder interactuar con los modelos. Finalmente, se obtuvieron los
resultados y plantearon las conclusiones.

\hypertarget{planteamiento-del-problema}{%
\section{Planteamiento del Problema}\label{planteamiento-del-problema}}

El Instituto Colombiano de Bienestar Familiar es una entidad que trabaja
por la prevención y protección integral de la primera infancia, la
niñez, la adolescencia y el bienestar general de las familias en
Colombia, llegando a millones de colombianos mediante sus programas,
estrategias y servicios de atención. En el marco de los objetivos de
esta institución se encontró que el ICBF actualmente no cuenta con una
herramienta para conocer en prospectiva, y de forma adecuada y efectiva
la satisfacción general de vida tanto de niños como de adultos en la
tercera edad. Es para ellos de vital importancia conocer esta
información pues es un indicador fundamental a tener en cuenta a la hora
de crear programas preventivos y de protección que tienen como objetivo
el mejoramiento de vida de la población destinataria. Por esto se busca
implementar en el ICBF tanto los 3 sub-modelos como el modelo de
satisfacción general, para que sea usado por la institución en pro de
mejorar futuros planeamientos en todo proyecto social que involucre
niños y adultos de la tercera edad como población objetivo.

\hypertarget{justificaciuxf3n}{%
\section{Justificación}\label{justificaciuxf3n}}

\hypertarget{niuxf1os}{%
\subsection{Niños}\label{niuxf1os}}

En primera instancia se planteó tomar a los niños en dos grupos, uno
como aquellos pertenecientes a la primera infancia (0 a 5 años) y otro
con aquellos niños con edad entre 6 y 12 años. Sin embargo, luego se
decidio que tomariamos como niño la definicion integrada en el codigo de
infancia y adolesencia, donde se expone que ``Para todos los efectos de
esta ley son sujetos titulares de derechos todas las personas menores de
18 años. Sin perjuicio de lo establecido en el artículo 34 del Código
Civil, se entiende por niño o niña las personas entre los 0 y los 12
años, y por adolescente las personas entre 12 y 18 años de
edad''(Articulo 3).

Una vez adoptada esta definicion se analizo cuantas observaciones de la
ECV cumplian esta condicion, resultando en un total de 56128 niños.

\hypertarget{abuelos}{%
\subsection{Abuelos}\label{abuelos}}

Para los abuelos, al igual que con los niños, se penso inicialmente en
tomar un rango de edad que tomara la definicion popular de este
colectivo, los adultos de la tercera edad (mayores de 60 años). Sin
embargo, luego planteamos que debiamos tener en cuenta cual es la
definicion literal de abuelo, y mediante un sistema de grafos logramos
determinar el numero de hombres y mujeres que tenian un nieto.
Filtrandolos por su rango edad.

\begin{table}[H]
\begin{center}
\begin{tabular}{|l|l|}
\hline
Mayor a 60 años & 1467 \\ \hline
Menor a 60 años & 1049 \\ \hline
\end{tabular}
\end{center}
\end{table}

Se observa que con la definicion inicial estabamos omitiendo un total de
1049 observaciones, ademas, se observa que los abuelos resgitrados en la
base de datos son relativamente pocos pues solo representan el x\% del
total de personas.

Ante esta situación tomamos la desicion de \ldots{}

\hypertarget{seleccion-de-variables}{%
\section{Seleccion de Variables}\label{seleccion-de-variables}}

Para predecir la satisfacción se tomaron en cuenta las evidencias
halladas en la bibliografía, donde se encontró que las variables
relacionadas con el nivel de satisfacción de una persona son las
siguientes:

\begin{itemize}
\tightlist
\item
  NIVEL\_DE\_EDUCACION
\item
  SEXO
\item
  SALUD\_AUTOPERCIBIDA
\item
  ESTADO\_CIVIL
\item
  ETNIA
\item
  INGRESO\_AUTOPERCIBIDO
\item
  SEGURIDAD\_AUTOPERCIBIDA
\item
  TRABAJO\_AUTOPERCIBIDO
\item
  SATISFACCION
\item
  I\_HOGAR
\item
  PERCAPITA
\item
  COND\_VIDA\_DEL\_HOGAR
\end{itemize}

El primer paso para verificar si las variables anteriores servirán para
ajustar algún modelo es hacer un análisis descriptivo de cada una
agrupando los niños y abuelos.

\hypertarget{modelos-predictivos}{%
\section{Modelos Predictivos}\label{modelos-predictivos}}

Inicialmente, se intentó englobar en un modelo a los abuelos y niños con
el fin de predecir la satisfacción. Sin embargo, como lo ilustra la
tabla 1, las variables objetivos que se seleccionaron no fueron
respondidas, en su mayoría, por niños. Este comportamiento se asemeja
con los resultados encontrados en \cite{b7}, donde se puede observar que
los abuelos y niños tienen diferentes definiciones de satisfacción y,
por ende, diferentes factores que la influyen. Por esta razón, se
decidió trabajar de forma independiente los modelos para los niños y
abuelos.

\hypertarget{modelos-predictivos-en-abuelos}{%
\subsection{Modelos Predictivos en
Abuelos}\label{modelos-predictivos-en-abuelos}}

\hypertarget{satisfacciuxf3n-de-la-vida-en-general}{%
\subsubsection{Satisfacción de la Vida en
General}\label{satisfacciuxf3n-de-la-vida-en-general}}

\hypertarget{anuxe1lisis-descriptivo}{%
\paragraph{Análisis Descriptivo}\label{anuxe1lisis-descriptivo}}

\hypertarget{conteo-de-respuestas}{%
\subparagraph{Conteo de respuestas}\label{conteo-de-respuestas}}

\begin{table}[H]
\centering
\resizebox{\linewidth}{!}{
\begin{tabular}{l|r|r|r|r|r|r|r}
\hline
Clasificación & ID\_Persona & NIVEL\_DE\_EDUCACION & SEXO & SALUD\_AUTOPERCIBIDA & ESTADO\_CIVIL & ETNIA & INGRESO\_AUTOPERCIBIDO\\
\hline
3ra\_edad & 37721 & 37721 & 37721 & 37721 & 37721 & 37721 & 37721\\
\hline
niño & 32576 & 32576 & 32576 & 32576 & 14464 & 32576 & 0\\
\hline
\end{tabular}}
\end{table}

\begin{table}[H]
\centering
\resizebox{\linewidth}{!}{
\begin{tabular}{r|r|r|r|r|r|r}
\hline
SEGURIDAD\_AUTOPERCIBIDA & TRABAJO\_AUTOPERCIBIDO & SATISFACCION & ETNIA.1 & P2059 & I\_HOGAR & PERCAPITA\\
\hline
37721 & 37721 & 37721 & 37721 & 37721 & 37721 & 37721\\
\hline
0 & 0 & 0 & 32576 & 32576 & 32576 & 32576\\
\hline
\end{tabular}}
\end{table}

Según la tabla anterior las variables que preguntan directamente por la
satisfacción no son respondidas por los niños, por lo tanto se decide
abordar y desarrollar modelos por separado para niños y abuelos.

\hypertarget{matriz-de-correlaciones}{%
\subparagraph{Matriz de Correlaciones}\label{matriz-de-correlaciones}}

Para empezar la selección de posibles variables objetivo y predictoras
de las escogidas anteriormente se desarrolla el siguiente gráfico de
correlación:

\includegraphics{InformeTecnico_files/figure-latex/correlación_satisfacción -1.pdf}
Se evalúa la posibilidad de establecer como posible variable objetivo la
satisfacción, esta tiene una buena correlación con las variables de
satisfacción pero no con el resto, por lo tanto es posible que no tenga
mucho sentido predecir la satisfacción general de un abuelo a partir de
todas estas variables. Con base en lo anterior se evalúa la
implementación de varios modelos, uno por cada variable de satisfacción.

\hypertarget{knn}{%
\paragraph{KNN}\label{knn}}

\begin{verbatim}
##    Y KNN_Predict
## 2  8          10
## 4  4          10
## 11 6          10
## 17 9          10
## 18 7          10
## 20 8          10
\end{verbatim}

\hypertarget{regresion-con-lm-y-glm}{%
\paragraph{Regresion con lm y glm}\label{regresion-con-lm-y-glm}}

\begin{verbatim}
## 
## Call:
## lm(formula = Y ~ ., data = train)
## 
## Residuals:
##     Min      1Q  Median      3Q     Max 
## -9.7221 -0.7421  0.0334  0.7687  7.2361 
## 
## Coefficients:
##              Estimate Std. Error t value Pr(>|t|)    
## (Intercept)  3.882886   0.059204  65.585  < 2e-16 ***
## X1           0.291916   0.005082  57.447  < 2e-16 ***
## X2           0.206558   0.005097  40.528  < 2e-16 ***
## X3           0.015651   0.005008   3.125  0.00178 ** 
## X4          -0.291487   0.014676 -19.862  < 2e-16 ***
## X5           0.137483   0.004334  31.720  < 2e-16 ***
## ---
## Signif. codes:  0 '***' 0.001 '**' 0.01 '*' 0.05 '.' 0.1 ' ' 1
## 
## Residual standard error: 1.437 on 28279 degrees of freedom
## Multiple R-squared:  0.4074, Adjusted R-squared:  0.4073 
## F-statistic:  3889 on 5 and 28279 DF,  p-value: < 2.2e-16
\end{verbatim}

\begin{verbatim}
##    Y Y_Predict
## 2  8  7.576154
## 4  4  7.471508
## 11 6  8.276218
## 17 9  8.620259
## 18 7  6.114202
## 20 8  7.915893
\end{verbatim}

\begin{verbatim}
##       1 
## 4.96325
\end{verbatim}

\hypertarget{satisfacciuxf3n-en-la-salud}{%
\subsubsection{Satisfacción en la
Salud}\label{satisfacciuxf3n-en-la-salud}}

Según (PONER BIBLIOGRAFÍA) se seleccionan las siguientes variables como
posibles predictores de la satisfacción de la Salud de los abuelos:

\begin{itemize}
\tightlist
\item
  AFILIADO (p6090)
\item
  PAGO\_EPS (p8551)
\item
  CALIDAD\_PRESTADOR (p6181)
\item
  ESTADO\_SALUD (p6127)
\item
  TIPO\_PAGO (p6115)
\item
  REGIMEN (P6100)
\item
  ENFERMEDAD\_CRONICA (P1930)
\end{itemize}

\hypertarget{anuxe1lisis-descriptivo-1}{%
\paragraph{Análisis Descriptivo}\label{anuxe1lisis-descriptivo-1}}

\begin{verbatim}
##    Min. 1st Qu.  Median    Mean 3rd Qu.    Max. 
##  -1.000   0.000   2.000   1.409   2.000   3.000
\end{verbatim}

\hypertarget{matriz-de-correlaciones-1}{%
\subparagraph{Matriz de Correlaciones}\label{matriz-de-correlaciones-1}}

\includegraphics{InformeTecnico_files/figure-latex/unnamed-chunk-9-1.pdf}

\hypertarget{regresion-con-lm}{%
\paragraph{Regresion con lm}\label{regresion-con-lm}}

\begin{verbatim}
##    SATISFACCION CALIDAD_EPS ESTADO_SALUD ESTRATO REGIMEN ENFERMEDAD_CRONICA
## 1             8           2            2       3       0                  2
## 3            10           2            2       3       0                  2
## 5             7           2            3       3       0                  2
## 11            1           1            4       3       2                  1
## 26            3           3            3       3       0                  1
## 29            4           2            3       3       2                  1
\end{verbatim}

\begin{verbatim}
## 
## Call:
## lm(formula = SATISFACCION ~ ., data = train_salud)
## 
## Residuals:
##     Min      1Q  Median      3Q     Max 
## -9.1444 -1.0243  0.1668  1.1887  5.7506 
## 
## Coefficients:
##                     Estimate Std. Error t value Pr(>|t|)    
## (Intercept)        10.420753   0.079567 130.968  < 2e-16 ***
## CALIDAD_EPS        -0.157173   0.022567  -6.965 3.36e-12 ***
## ESTADO_SALUD       -1.426687   0.019721 -72.343  < 2e-16 ***
## ESTRATO             0.021954   0.008358   2.627  0.00863 ** 
## REGIMEN            -0.157869   0.012427 -12.704  < 2e-16 ***
## ENFERMEDAD_CRONICA  0.300661   0.024138  12.456  < 2e-16 ***
## ---
## Signif. codes:  0 '***' 0.001 '**' 0.01 '*' 0.05 '.' 0.1 ' ' 1
## 
## Residual standard error: 1.848 on 27607 degrees of freedom
## Multiple R-squared:  0.2271, Adjusted R-squared:  0.2269 
## F-statistic:  1622 on 5 and 27607 DF,  p-value: < 2.2e-16
\end{verbatim}

\begin{verbatim}
##    SATISFACCION Y_Predict
## 2             7  7.920215
## 4             7  6.192868
## 25            9  7.763042
## 41            8  7.604477
## 59            4  6.493529
## 63           10  7.761650
\end{verbatim}

\begin{verbatim}
## 'data.frame':    27613 obs. of  6 variables:
##  $ SATISFACCION      : int  8 10 7 1 3 4 9 5 6 6 ...
##  $ CALIDAD_EPS       : int  2 2 2 1 3 2 2 2 2 2 ...
##  $ ESTADO_SALUD      : int  2 2 3 4 3 3 2 3 2 2 ...
##  $ ESTRATO           : int  3 3 3 3 3 3 3 3 3 3 ...
##  $ REGIMEN           : num  0 0 0 2 0 2 2 0 0 2 ...
##  $ ENFERMEDAD_CRONICA: int  2 2 2 1 1 1 2 2 2 2 ...
\end{verbatim}

\hypertarget{satisfacciuxf3n-sobre-el-nivel-de-seguridad}{%
\subsubsection{Satisfacción sobre el Nivel de
Seguridad}\label{satisfacciuxf3n-sobre-el-nivel-de-seguridad}}

Según (PONER BIBLIOGRAFÍA) se seleccionan las siguientes variables como
posibles predictores de la satisfacción de la seguridad de los abuelos:

\begin{itemize}
\tightlist
\item
  ESTADO\_CIVIL (P5502)
\item
  SEXO (P6020´)
\item
  ESTRATO (P8520S1A1)
\item
  INGRESOS (P8624) -\textgreater{} por fuera por muchos nans
\item
  HORAS\_LABORALES (P415) -\textgreater{} fuera
\item
  NIVEL\_DE\_SEGURIDAD (P9010)
\item
  CONDICIONES\_DE\_VIDA\_HOGAR (P9030)
\item
  ES\_CAMPESINO (P2059)
\item
  ES\_POBRE (P5230) -\textgreater{} fuera
\end{itemize}

\hypertarget{anuxe1lisis-descriptivo-2}{%
\paragraph{Análisis Descriptivo}\label{anuxe1lisis-descriptivo-2}}

\hypertarget{matriz-de-correlaciones-2}{%
\subparagraph{Matriz de Correlaciones}\label{matriz-de-correlaciones-2}}

\includegraphics{InformeTecnico_files/figure-latex/unnamed-chunk-12-1.pdf}
\#\#\#\# Regresion con lm

\begin{verbatim}
## 
## Call:
## lm(formula = SATISFACCION ~ ., data = train_seguridad)
## 
## Residuals:
##     Min      1Q  Median      3Q     Max 
## -8.1538 -0.9253  0.2007  1.3096  4.7036 
## 
## Coefficients:
##                            Estimate Std. Error t value Pr(>|t|)    
## (Intercept)               10.246177   0.089403 114.606   <2e-16 ***
## ESTADO_CIVIL               0.044585   0.007275   6.128    9e-10 ***
## SEXO                       0.006390   0.022931   0.279    0.780    
## ESTRATO                   -0.007751   0.008366  -0.927    0.354    
## NIVEL_DE_SEGURIDAD        -1.548062   0.036358 -42.578   <2e-16 ***
## CONDICIONES_DE_VIDA_HOGAR -0.486674   0.018827 -25.850   <2e-16 ***
## ES_CAMPESINO              -0.064281   0.024968  -2.575    0.010 *  
## ---
## Signif. codes:  0 '***' 0.001 '**' 0.01 '*' 0.05 '.' 0.1 ' ' 1
## 
## Residual standard error: 1.902 on 27748 degrees of freedom
## Multiple R-squared:  0.09431,    Adjusted R-squared:  0.09411 
## F-statistic: 481.6 on 6 and 27748 DF,  p-value: < 2.2e-16
\end{verbatim}

\begin{verbatim}
##    SATISFACCION Y_Predict
## 2            10  7.271007
## 4            10  7.846852
## 19            7  7.846852
## 35            5  7.808657
## 36            8  7.764072
## 55            7  7.719486
\end{verbatim}

\hypertarget{satisfacciuxf3n-en-el-trabajo}{%
\subsubsection{Satisfacción en el
Trabajo}\label{satisfacciuxf3n-en-el-trabajo}}

Según (PONER BIBLIOGRAFÍA) se seleccionan las siguientes variables como
posibles predictores de la satisfacción en el trabajo de los abuelos:

\begin{itemize}
\tightlist
\item
  SEXO (P6020´)
\item
  CARGO (P6435)
\item
  TIENE\_CONTRATO (P6440)
\item
  TIPO\_CONTRATO (P6460)
\item
  SALARIO (P8624)
\item
  HORAS\_LABORALES (P415)
\item
  RECIBIO\_PRIMAS (P8631)
\item
  RECIBIO\_PENSIONES (P8642)
\end{itemize}

\hypertarget{anuxe1lisis-descriptivo-3}{%
\paragraph{Análisis Descriptivo}\label{anuxe1lisis-descriptivo-3}}

\hypertarget{datos-faltantes}{%
\paragraph{Datos Faltantes}\label{datos-faltantes}}

Para este dataframe se tiene la siguiente cantidad de abuelos:

\begin{verbatim}
##   Cantidad de Abuelos
## 1               37013
\end{verbatim}

Sin embargo, si se observa la cantidad de abuelos que respondieron a las
preguntas seleccionadas:
\includegraphics{InformeTecnico_files/figure-latex/unnamed-chunk-16-1.pdf}

se puede determinar que la mayoría de estos no respondieron a las
preguntas que se les hicieron sobre el trabajo. Este mismo procedimiento
se repitió con variables diferentes, pero no se obtuvieron resultados
distintos a los presentados. Por tal motivo, se decide no realizar un
modelo de predicción para la satisfacción del trabajo en los abuelos. En
primera instancia se pensó que este comportamiento se debía a que la
mayoría de los abuelos estaban pensionados, pero si se observan los
abuelos pensionados:

\includegraphics{InformeTecnico_files/figure-latex/unnamed-chunk-17-1.pdf}

la gran mayoría de estos respondieron que no recibían algún tipo de
pensión (2). Por tanto, se puede inferir que la mayoría de los abuelos
no trabajan y tampoco reciben pensión, es decir, viven dependientes de
sus familiares.

\hypertarget{modelos-predictivos-en-niuxf1os}{%
\subsection{Modelos Predictivos en
Niños}\label{modelos-predictivos-en-niuxf1os}}

\begin{Shaded}
\begin{Highlighting}[]
\CommentTok{\# los datos:}
\NormalTok{ninos }\OtherTok{\textless{}{-}} \FunctionTok{read.csv2}\NormalTok{(}\StringTok{"Datos/datos\_ninos.csv"}\NormalTok{, }\AttributeTok{header=}\ConstantTok{TRUE}\NormalTok{, }\AttributeTok{dec=}\StringTok{"."}\NormalTok{, }\AttributeTok{encoding=}\StringTok{"UTF{-}8"}\NormalTok{)}
\end{Highlighting}
\end{Shaded}

\hypertarget{correlaciones}{%
\subsubsection{Correlaciones}\label{correlaciones}}

\begin{Shaded}
\begin{Highlighting}[]
\NormalTok{corr\_ninos }\OtherTok{\textless{}{-}}\NormalTok{ ninos }\SpecialCharTok{\%\textgreater{}\%} \FunctionTok{select}\NormalTok{(SATISFACCION, ESTADO\_SALUD, ACTIVIDADES, ETNIA, }
\NormalTok{                               CONDIC\_VIDA\_HOGAR, INCRESOS\_AUTOPERCIBIDOS\_HOGAR,}
\NormalTok{                               PERCAPITA, CANT\_PERSONAS\_HOGAR, ESCUELA\_OFICIAL,}
\NormalTok{                               VALOR\_BECA, ORIGEN\_SUBSIDIO\_EDUC, UBICACION\_ESCUELA, }
\NormalTok{                               TRANSPORTE\_ESCUELA, TIEMPO\_TRANSPORTE\_ESC, }
\NormalTok{                               EDUCACION\_PADRES ,ACTIVIDAD\_ULT\_SEMANA, }
\NormalTok{                               LUGAR\_TRABAJO, ENFERMEDAD\_CRONICA, INCAP)}

\NormalTok{M }\OtherTok{=} \FunctionTok{cor}\NormalTok{(corr\_ninos, }\AttributeTok{use =} \StringTok{"pairwise.complete.obs"}\NormalTok{)}

\NormalTok{M[}\FunctionTok{is.na}\NormalTok{(M)]}\OtherTok{=}\DecValTok{0}

\FunctionTok{ggcorrplot}\NormalTok{(M, }\AttributeTok{hc.order =} \ConstantTok{TRUE}\NormalTok{, }\AttributeTok{type =} \StringTok{"lower"}\NormalTok{, }\AttributeTok{lab =} \ConstantTok{TRUE}\NormalTok{,}\AttributeTok{tl.cex =} \DecValTok{6}\NormalTok{, }
            \AttributeTok{pch.col =} \StringTok{"red"}\NormalTok{, }\AttributeTok{lab\_size =} \DecValTok{2}\NormalTok{)}
\end{Highlighting}
\end{Shaded}

\includegraphics{InformeTecnico_files/figure-latex/unnamed-chunk-19-1.pdf}

Los datos para niños:

\begin{Shaded}
\begin{Highlighting}[]
\NormalTok{df\_ninos }\OtherTok{\textless{}{-}}\NormalTok{ ninos }\SpecialCharTok{\%\textgreater{}\%} \FunctionTok{select}\NormalTok{(EDAD, CANT\_PERSONAS\_HOGAR, INCAP, SATISFACCION, }
\NormalTok{                             PERCAPITA, INCRESOS\_AUTOPERCIBIDOS\_HOGAR, }
\NormalTok{                             UBICACION\_ESCUELA, EDUCACION\_PADRES,}
\NormalTok{                             CONDIC\_VIDA\_HOGAR)}

\NormalTok{df\_ninos[}\StringTok{\textquotesingle{}ETNIA\textquotesingle{}}\NormalTok{] }\OtherTok{=} \FunctionTok{as.factor}\NormalTok{(ninos[,}\StringTok{\textquotesingle{}ETNIA\textquotesingle{}}\NormalTok{])}

\NormalTok{df\_ninos[}\StringTok{\textquotesingle{}VIVE\_CON\_PADRE\textquotesingle{}}\NormalTok{] }\OtherTok{=} \FunctionTok{factor}\NormalTok{(ninos[,}\StringTok{\textquotesingle{}VIVE\_CON\_PADRE\textquotesingle{}}\NormalTok{], }\AttributeTok{labels =} \FunctionTok{c}\NormalTok{(}\ConstantTok{TRUE}\NormalTok{, }
                                                                         \ConstantTok{FALSE}\NormalTok{, }
                                                                         \StringTok{\textquotesingle{}Muerto\textquotesingle{}}\NormalTok{))}

\NormalTok{df\_ninos[}\StringTok{\textquotesingle{}VIVE\_CON\_MADRE\textquotesingle{}}\NormalTok{] }\OtherTok{=} \FunctionTok{factor}\NormalTok{(ninos[,}\StringTok{\textquotesingle{}VIVE\_CON\_MADRE\textquotesingle{}}\NormalTok{], }\AttributeTok{labels =} \FunctionTok{c}\NormalTok{(}\ConstantTok{TRUE}\NormalTok{, }
                                                                         \ConstantTok{FALSE}\NormalTok{, }
                                                                         \StringTok{\textquotesingle{}Muerto\textquotesingle{}}\NormalTok{))}

\NormalTok{df\_ninos[}\StringTok{\textquotesingle{}ESCUELA\_OFICIAL\textquotesingle{}}\NormalTok{] }\OtherTok{=} \FunctionTok{factor}\NormalTok{(ninos[,}\StringTok{\textquotesingle{}ESCUELA\_OFICIAL\textquotesingle{}}\NormalTok{], }\AttributeTok{labels =} \FunctionTok{c}\NormalTok{(}\StringTok{\textquotesingle{}Oficial\textquotesingle{}}\NormalTok{, }
                                                              \StringTok{\textquotesingle{}conSubstito\textquotesingle{}}\NormalTok{, }
                                                              \StringTok{\textquotesingle{}SinSubstito\textquotesingle{}}\NormalTok{))}

\NormalTok{df\_ninos[}\StringTok{\textquotesingle{}TRANSPORTE\_ESCUELA\textquotesingle{}}\NormalTok{] }\OtherTok{=} \FunctionTok{addNA}\NormalTok{(}\FunctionTok{factor}\NormalTok{(ninos[,}\StringTok{\textquotesingle{}TRANSPORTE\_ESCUELA\textquotesingle{}}\NormalTok{], }
                                        \AttributeTok{labels =} \FunctionTok{c}\NormalTok{(}\StringTok{\textquotesingle{}Carro\textquotesingle{}}\NormalTok{, }\StringTok{\textquotesingle{}escolar\textquotesingle{}}\NormalTok{, }\StringTok{\textquotesingle{}público\textquotesingle{}}\NormalTok{,}
                                                   \StringTok{\textquotesingle{}pie\textquotesingle{}}\NormalTok{, }\StringTok{\textquotesingle{}Bicicleta\textquotesingle{}}\NormalTok{, }\StringTok{\textquotesingle{}Caballo\textquotesingle{}}\NormalTok{,}
                                                   \StringTok{\textquotesingle{}canoa\textquotesingle{}}\NormalTok{, }\StringTok{\textquotesingle{}Otro\textquotesingle{}}\NormalTok{)))}

\NormalTok{df\_ninos[}\StringTok{\textquotesingle{}ENFERMEDAD\_CRONICA\textquotesingle{}}\NormalTok{] }\OtherTok{=}  \FunctionTok{addNA}\NormalTok{(}\FunctionTok{factor}\NormalTok{(ninos[,}\StringTok{\textquotesingle{}ENFERMEDAD\_CRONICA\textquotesingle{}}\NormalTok{], }
                                               \AttributeTok{labels =} \FunctionTok{c}\NormalTok{(}\ConstantTok{TRUE}\NormalTok{, }\ConstantTok{FALSE}\NormalTok{)))}

\NormalTok{df\_ninos[}\StringTok{\textquotesingle{}ACTIVIDAD\_ULT\_SEMANA\textquotesingle{}}\NormalTok{] }\OtherTok{=} \FunctionTok{addNA}\NormalTok{(}\FunctionTok{factor}\NormalTok{(ninos[,}\StringTok{\textquotesingle{}ACTIVIDAD\_ULT\_SEMANA\textquotesingle{}}\NormalTok{], }
                                          \AttributeTok{labels =} \FunctionTok{c}\NormalTok{( }\StringTok{\textquotesingle{}Trabajando\textquotesingle{}}\NormalTok{, }\StringTok{\textquotesingle{}Buscando\textquotesingle{}}\NormalTok{,}
                                                      \StringTok{\textquotesingle{}Estudiando\textquotesingle{}}\NormalTok{, }\StringTok{\textquotesingle{}Oficios\_hogar\textquotesingle{}}\NormalTok{, }
                                                      \StringTok{\textquotesingle{}Incapacitado trabajar\textquotesingle{}}\NormalTok{, }\StringTok{\textquotesingle{}Otra\textquotesingle{}}\NormalTok{)))}

\NormalTok{df\_ninos[}\StringTok{\textquotesingle{}LUGAR\_TRABAJO\textquotesingle{}}\NormalTok{] }\OtherTok{=} \FunctionTok{addNA}\NormalTok{(}\FunctionTok{factor}\NormalTok{(ninos[,}\StringTok{\textquotesingle{}LUGAR\_TRABAJO\textquotesingle{}}\NormalTok{], }
                                         \AttributeTok{labels =} \FunctionTok{c}\NormalTok{(}\StringTok{\textquotesingle{}no trabajo\textquotesingle{}}\NormalTok{, }\StringTok{\textquotesingle{}la vivienda\textquotesingle{}}\NormalTok{,}
                                                    \StringTok{\textquotesingle{}otra vivienda\textquotesingle{}}\NormalTok{, }\StringTok{\textquotesingle{}Puerta\textquotesingle{}}\NormalTok{, }
                                                    \StringTok{\textquotesingle{}calle\textquotesingle{}}\NormalTok{, }\StringTok{\textquotesingle{}oficina\textquotesingle{}}\NormalTok{, }\StringTok{\textquotesingle{}campo\textquotesingle{}}\NormalTok{,}
                                                    \StringTok{\textquotesingle{}obra\textquotesingle{}}\NormalTok{)))}

\FunctionTok{head}\NormalTok{(df\_ninos)}
\end{Highlighting}
\end{Shaded}

\begin{verbatim}
##   EDAD CANT_PERSONAS_HOGAR INCAP SATISFACCION PERCAPITA
## 1    7                   6    32           31  694500.0
## 2   10                   4    32           25  265083.3
## 3    8                   4    32           34  300000.0
## 4    7                   2    32           25  759078.5
## 5    6                   6    32           31  954500.0
## 6    7                   5    32           31  600000.0
##   INCRESOS_AUTOPERCIBIDOS_HOGAR UBICACION_ESCUELA EDUCACION_PADRES
## 1                             2                 1                2
## 2                             2                 1                4
## 3                             2                 1               NA
## 4                             1                 1                3
## 5                             2                 1               NA
## 6                             2                 1               NA
##   CONDIC_VIDA_HOGAR ETNIA VIVE_CON_PADRE VIVE_CON_MADRE ESCUELA_OFICIAL
## 1                 2     6          FALSE          FALSE         Oficial
## 2                 2     6          FALSE           TRUE         Oficial
## 3                 2     6           TRUE           TRUE     SinSubstito
## 4                 2     6          FALSE           TRUE     SinSubstito
## 5                 3     6           TRUE           TRUE     SinSubstito
## 6                 2     6           TRUE           TRUE     SinSubstito
##   TRANSPORTE_ESCUELA ENFERMEDAD_CRONICA ACTIVIDAD_ULT_SEMANA LUGAR_TRABAJO
## 1                pie              FALSE           Estudiando    no trabajo
## 2              Carro              FALSE           Estudiando    no trabajo
## 3            escolar              FALSE           Estudiando    no trabajo
## 4              Carro              FALSE           Estudiando    no trabajo
## 5                pie              FALSE           Estudiando    no trabajo
## 6                pie              FALSE           Estudiando    no trabajo
\end{verbatim}

\hypertarget{resultados-del-entrenamiento}{%
\subsection{Resultados del
entrenamiento}\label{resultados-del-entrenamiento}}

Separación de los datos en entrenamiento y prueba

\begin{Shaded}
\begin{Highlighting}[]
\NormalTok{datos1 }\OtherTok{\textless{}{-}} \FunctionTok{sample}\NormalTok{(}\DecValTok{2}\NormalTok{, }\FunctionTok{nrow}\NormalTok{(df\_ninos),}
                   \AttributeTok{replace =} \ConstantTok{TRUE}\NormalTok{,}
                    \AttributeTok{prob =} \FunctionTok{c}\NormalTok{(}\FloatTok{0.75}\NormalTok{, }\FloatTok{0.25}\NormalTok{))}

\NormalTok{train }\OtherTok{\textless{}{-}}\NormalTok{ df\_ninos[datos1 }\SpecialCharTok{==} \DecValTok{1}\NormalTok{,]}
\NormalTok{test }\OtherTok{\textless{}{-}}\NormalTok{ df\_ninos[datos1 }\SpecialCharTok{==} \DecValTok{2}\NormalTok{,]}
\end{Highlighting}
\end{Shaded}

Modelo de árbol de decisión

\begin{Shaded}
\begin{Highlighting}[]
\NormalTok{modelo }\OtherTok{\textless{}{-}} \FunctionTok{ctree}\NormalTok{(}\AttributeTok{formula =}\NormalTok{ SATISFACCION }\SpecialCharTok{\textasciitilde{}}\NormalTok{ ., }\AttributeTok{data=}\NormalTok{train, }
                \AttributeTok{controls =} \FunctionTok{ctree\_control}\NormalTok{(}\AttributeTok{mincriterion =} \FloatTok{0.7}\NormalTok{))}
\end{Highlighting}
\end{Shaded}

\begin{Shaded}
\begin{Highlighting}[]
\FunctionTok{png}\NormalTok{(}\AttributeTok{file =} \StringTok{"decision\_tree.png"}\NormalTok{, }\AttributeTok{width =} \DecValTok{3000}\NormalTok{, }\AttributeTok{height =} \DecValTok{1500}\NormalTok{, )}
\FunctionTok{plot}\NormalTok{(modelo)}

\FunctionTok{dev.off}\NormalTok{()}
\end{Highlighting}
\end{Shaded}

\begin{verbatim}
## pdf 
##   2
\end{verbatim}

\begin{Shaded}
\begin{Highlighting}[]
\NormalTok{modelo}
\end{Highlighting}
\end{Shaded}

\begin{verbatim}
## 
##   Conditional inference tree with 100 terminal nodes
## 
## Response:  SATISFACCION 
## Inputs:  EDAD, CANT_PERSONAS_HOGAR, INCAP, PERCAPITA, INCRESOS_AUTOPERCIBIDOS_HOGAR, UBICACION_ESCUELA, EDUCACION_PADRES, CONDIC_VIDA_HOGAR, ETNIA, VIVE_CON_PADRE, VIVE_CON_MADRE, ESCUELA_OFICIAL, TRANSPORTE_ESCUELA, ENFERMEDAD_CRONICA, ACTIVIDAD_ULT_SEMANA, LUGAR_TRABAJO 
## Number of observations:  23479 
## 
## 1) INCRESOS_AUTOPERCIBIDOS_HOGAR <= 1; criterion = 1, statistic = 405.893
##   2) UBICACION_ESCUELA <= 2; criterion = 1, statistic = 162.538
##     3) ENFERMEDAD_CRONICA == {FALSE}; criterion = 1, statistic = 97.321
##       4) TRANSPORTE_ESCUELA == {Carro, pie, Bicicleta}; criterion = 1, statistic = 85.09
##         5) EDUCACION_PADRES <= 1; criterion = 1, statistic = 32.569
##           6) ESCUELA_OFICIAL == {SinSubstito}; criterion = 1, statistic = 36.538
##             7) TRANSPORTE_ESCUELA == {Carro, Bicicleta}; criterion = 0.957, statistic = 11.795
##               8)*  weights = 56 
##             7) TRANSPORTE_ESCUELA == {pie}
##               9)*  weights = 92 
##           6) ESCUELA_OFICIAL == {Oficial, conSubstito}
##             10) ETNIA == {3, 6}; criterion = 1, statistic = 29.552
##               11) PERCAPITA <= 196000; criterion = 0.926, statistic = 8.77
##                 12) EDUCACION_PADRES <= 0; criterion = 0.997, statistic = 14.112
##                   13) UBICACION_ESCUELA <= 1; criterion = 0.985, statistic = 10.88
##                     14)*  weights = 183 
##                   13) UBICACION_ESCUELA > 1
##                     15) PERCAPITA <= 47916.67; criterion = 0.993, statistic = 12.467
##                       16)*  weights = 9 
##                     15) PERCAPITA > 47916.67
##                       17)*  weights = 42 
##                 12) EDUCACION_PADRES > 0
##                   18) VIVE_CON_PADRE == {TRUE}; criterion = 0.974, statistic = 12.783
##                     19) CANT_PERSONAS_HOGAR <= 5; criterion = 0.744, statistic = 5.568
##                       20) INCAP <= 30; criterion = 0.934, statistic = 8.164
##                         21)*  weights = 9 
##                       20) INCAP > 30
##                         22)*  weights = 585 
##                     19) CANT_PERSONAS_HOGAR > 5
##                       23)*  weights = 325 
##                   18) VIVE_CON_PADRE == {FALSE, Muerto}
##                     24)*  weights = 376 
##               11) PERCAPITA > 196000
##                 25) INCAP <= 30; criterion = 0.89, statistic = 7.814
##                   26)*  weights = 27 
##                 25) INCAP > 30
##                   27) UBICACION_ESCUELA <= 1; criterion = 0.83, statistic = 7.722
##                     28)*  weights = 1377 
##                   27) UBICACION_ESCUELA > 1
##                     29)*  weights = 361 
##             10) ETNIA == {1, 4, 5}
##               30) CONDIC_VIDA_HOGAR <= 1; criterion = 0.998, statistic = 15.054
##                 31)*  weights = 31 
##               30) CONDIC_VIDA_HOGAR > 1
##                 32) PERCAPITA <= 94583.33; criterion = 0.91, statistic = 13.266
##                   33) CONDIC_VIDA_HOGAR <= 2; criterion = 0.999, statistic = 15.838
##                     34) VIVE_CON_PADRE == {FALSE, Muerto}; criterion = 0.941, statistic = 11.147
##                       35)*  weights = 30 
##                     34) VIVE_CON_PADRE == {TRUE}
##                       36)*  weights = 115 
##                   33) CONDIC_VIDA_HOGAR > 2
##                     37) PERCAPITA <= 41250; criterion = 0.912, statistic = 12.069
##                       38)*  weights = 93 
##                     37) PERCAPITA > 41250
##                       39)*  weights = 132 
##                 32) PERCAPITA > 94583.33
##                   40) ESCUELA_OFICIAL == {Oficial}; criterion = 0.999, statistic = 16.238
##                     41)*  weights = 814 
##                   40) ESCUELA_OFICIAL == {conSubstito}
##                     42)*  weights = 11 
##         5) EDUCACION_PADRES > 1
##           43) EDUCACION_PADRES <= 2; criterion = 0.98, statistic = 10.431
##             44) EDAD <= 10; criterion = 0.964, statistic = 9.295
##               45) CANT_PERSONAS_HOGAR <= 3; criterion = 0.744, statistic = 5.659
##                 46)*  weights = 270 
##               45) CANT_PERSONAS_HOGAR > 3
##                 47)*  weights = 620 
##             44) EDAD > 10
##               48)*  weights = 358 
##           43) EDUCACION_PADRES > 2
##             49)*  weights = 161 
##       4) TRANSPORTE_ESCUELA == {escolar, público, Caballo, canoa, Otro, NA}
##         50) ETNIA == {2, 3, 4, 5, 6}; criterion = 1, statistic = 30.391
##           51) INCAP <= 30; criterion = 1, statistic = 19.12
##             52) EDUCACION_PADRES <= 0; criterion = 0.706, statistic = 6.028
##               53)*  weights = 11 
##             52) EDUCACION_PADRES > 0
##               54)*  weights = 35 
##           51) INCAP > 30
##             55) CANT_PERSONAS_HOGAR <= 7; criterion = 0.991, statistic = 14.329
##               56) CONDIC_VIDA_HOGAR <= 1; criterion = 0.996, statistic = 15.057
##                 57)*  weights = 63 
##               56) CONDIC_VIDA_HOGAR > 1
##                 58) CANT_PERSONAS_HOGAR <= 2; criterion = 0.958, statistic = 15.301
##                   59) TRANSPORTE_ESCUELA == {escolar, canoa, NA}; criterion = 0.771, statistic = 13.922
##                     60)*  weights = 129 
##                   59) TRANSPORTE_ESCUELA == {público, Caballo, Otro}
##                     61)*  weights = 27 
##                 58) CANT_PERSONAS_HOGAR > 2
##                   62) UBICACION_ESCUELA <= 1; criterion = 0.756, statistic = 13.697
##                     63) TRANSPORTE_ESCUELA == {escolar, público, Otro, NA}; criterion = 0.962, statistic = 18.461
##                       64)*  weights = 1809 
##                     63) TRANSPORTE_ESCUELA == {Caballo, canoa}
##                       65) ETNIA == {5}; criterion = 0.903, statistic = 7.336
##                         66)*  weights = 20 
##                       65) ETNIA == {6}
##                         67)*  weights = 17 
##                   62) UBICACION_ESCUELA > 1
##                     68)*  weights = 425 
##             55) CANT_PERSONAS_HOGAR > 7
##               69) PERCAPITA <= 288611.1; criterion = 0.987, statistic = 11.213
##                 70)*  weights = 104 
##               69) PERCAPITA > 288611.1
##                 71)*  weights = 15 
##         50) ETNIA == {1}
##           72) CONDIC_VIDA_HOGAR <= 2; criterion = 0.808, statistic = 8.304
##             73)*  weights = 196 
##           72) CONDIC_VIDA_HOGAR > 2
##             74) PERCAPITA <= 71190.48; criterion = 0.997, statistic = 13.687
##               75) TRANSPORTE_ESCUELA == {canoa, NA}; criterion = 0.95, statistic = 13.774
##                 76)*  weights = 66 
##               75) TRANSPORTE_ESCUELA == {escolar, Otro}
##                 77)*  weights = 13 
##             74) PERCAPITA > 71190.48
##               78) PERCAPITA <= 410000; criterion = 0.954, statistic = 8.834
##                 79)*  weights = 106 
##               78) PERCAPITA > 410000
##                 80)*  weights = 8 
##     3) ENFERMEDAD_CRONICA == {TRUE}
##       81) INCAP <= 30; criterion = 0.975, statistic = 9.973
##         82)*  weights = 46 
##       81) INCAP > 30
##         83) ESCUELA_OFICIAL == {conSubstito, SinSubstito}; criterion = 0.938, statistic = 11.039
##           84)*  weights = 18 
##         83) ESCUELA_OFICIAL == {Oficial}
##           85)*  weights = 221 
##   2) UBICACION_ESCUELA > 2
##     86) ENFERMEDAD_CRONICA == {FALSE}; criterion = 1, statistic = 65.834
##       87) ETNIA == {4, 6}; criterion = 1, statistic = 42.019
##         88) EDUCACION_PADRES <= 1; criterion = 0.991, statistic = 11.884
##           89) PERCAPITA <= 56750; criterion = 0.71, statistic = 12.101
##             90) VIVE_CON_PADRE == {FALSE}; criterion = 0.851, statistic = 6.635
##               91)*  weights = 35 
##             90) VIVE_CON_PADRE == {TRUE}
##               92)*  weights = 51 
##           89) PERCAPITA > 56750
##             93)*  weights = 1590 
##         88) EDUCACION_PADRES > 1
##           94) LUGAR_TRABAJO == {no trabajo}; criterion = 0.985, statistic = 11.676
##             95)*  weights = 197 
##           94) LUGAR_TRABAJO == {NA}
##             96)*  weights = 25 
##       87) ETNIA == {1, 2, 5}
##         97) ESCUELA_OFICIAL == {SinSubstito}; criterion = 1, statistic = 25.669
##           98)*  weights = 18 
##         97) ESCUELA_OFICIAL == {Oficial, conSubstito}
##           99) EDUCACION_PADRES <= 0; criterion = 0.926, statistic = 15.627
##             100)*  weights = 170 
##           99) EDUCACION_PADRES > 0
##             101) ACTIVIDAD_ULT_SEMANA == {Estudiando, Otra, NA}; criterion = 0.856, statistic = 12.585
##               102)*  weights = 1279 
##             101) ACTIVIDAD_ULT_SEMANA == {Oficios_hogar}
##               103)*  weights = 54 
##     86) ENFERMEDAD_CRONICA == {TRUE}
##       104) PERCAPITA <= 352500; criterion = 0.731, statistic = 8.66
##         105)*  weights = 55 
##       104) PERCAPITA > 352500
##         106)*  weights = 10 
## 1) INCRESOS_AUTOPERCIBIDOS_HOGAR > 1
##   107) CONDIC_VIDA_HOGAR <= 1; criterion = 1, statistic = 280.184
##     108) INCRESOS_AUTOPERCIBIDOS_HOGAR <= 2; criterion = 0.997, statistic = 24.711
##       109) VIVE_CON_PADRE == {Muerto}; criterion = 0.919, statistic = 14.025
##         110)*  weights = 13 
##       109) VIVE_CON_PADRE == {TRUE, FALSE}
##         111)*  weights = 668 
##     108) INCRESOS_AUTOPERCIBIDOS_HOGAR > 2
##       112) CANT_PERSONAS_HOGAR <= 6; criterion = 0.898, statistic = 10.535
##         113)*  weights = 208 
##       112) CANT_PERSONAS_HOGAR > 6
##         114)*  weights = 12 
##   107) CONDIC_VIDA_HOGAR > 1
##     115) UBICACION_ESCUELA <= 1; criterion = 1, statistic = 132.107
##       116) PERCAPITA <= 373444.4; criterion = 1, statistic = 50.872
##         117) TRANSPORTE_ESCUELA == {Carro, público, pie, Otro, NA}; criterion = 1, statistic = 46.687
##           118) ENFERMEDAD_CRONICA == {FALSE}; criterion = 1, statistic = 22.085
##             119) ESCUELA_OFICIAL == {SinSubstito}; criterion = 0.731, statistic = 8.198
##               120) TRANSPORTE_ESCUELA == {Carro}; criterion = 0.899, statistic = 14.21
##                 121)*  weights = 33 
##               120) TRANSPORTE_ESCUELA == {público, pie, Otro, NA}
##                 122)*  weights = 123 
##             119) ESCUELA_OFICIAL == {Oficial, conSubstito}
##               123) PERCAPITA <= 133125; criterion = 0.822, statistic = 7.527
##                 124) VIVE_CON_MADRE == {TRUE}; criterion = 0.883, statistic = 7.119
##                   125)*  weights = 528 
##                 124) VIVE_CON_MADRE == {FALSE}
##                   126)*  weights = 71 
##               123) PERCAPITA > 133125
##                 127)*  weights = 2161 
##           118) ENFERMEDAD_CRONICA == {TRUE}
##             128) LUGAR_TRABAJO == {no trabajo}; criterion = 0.736, statistic = 7.655
##               129)*  weights = 60 
##             128) LUGAR_TRABAJO == {NA}
##               130)*  weights = 12 
##         117) TRANSPORTE_ESCUELA == {escolar, Bicicleta, Caballo, canoa}
##           131) PERCAPITA <= 185266.7; criterion = 1, statistic = 22.888
##             132) CONDIC_VIDA_HOGAR <= 2; criterion = 0.894, statistic = 7.277
##               133) ETNIA == {1, 5}; criterion = 0.955, statistic = 11.709
##                 134)*  weights = 13 
##               133) ETNIA == {6}
##                 135)*  weights = 76 
##             132) CONDIC_VIDA_HOGAR > 2
##               136)*  weights = 25 
##           131) PERCAPITA > 185266.7
##             137)*  weights = 155 
##       116) PERCAPITA > 373444.4
##         138) INCRESOS_AUTOPERCIBIDOS_HOGAR <= 2; criterion = 1, statistic = 26.122
##           139) ESCUELA_OFICIAL == {SinSubstito}; criterion = 0.991, statistic = 20.903
##             140)*  weights = 534 
##           139) ESCUELA_OFICIAL == {Oficial, conSubstito}
##             141) INCAP <= 30; criterion = 0.991, statistic = 16.048
##               142) INCAP <= 27; criterion = 0.952, statistic = 8.774
##                 143)*  weights = 9 
##               142) INCAP > 27
##                 144)*  weights = 49 
##             141) INCAP > 30
##               145) TRANSPORTE_ESCUELA == {escolar, Bicicleta, canoa}; criterion = 0.759, statistic = 17.047
##                 146) VIVE_CON_PADRE == {TRUE}; criterion = 0.858, statistic = 9.305
##                   147)*  weights = 140 
##                 146) VIVE_CON_PADRE == {FALSE, Muerto}
##                   148)*  weights = 103 
##               145) TRANSPORTE_ESCUELA == {Carro, público, pie, Otro, NA}
##                 149) EDAD <= 6; criterion = 0.953, statistic = 10.644
##                   150)*  weights = 283 
##                 149) EDAD > 6
##                   151) EDUCACION_PADRES <= 0; criterion = 0.905, statistic = 10.419
##                     152)*  weights = 142 
##                   151) EDUCACION_PADRES > 0
##                     153)*  weights = 1713 
##         138) INCRESOS_AUTOPERCIBIDOS_HOGAR > 2
##           154)*  weights = 290 
##     115) UBICACION_ESCUELA > 1
##       155) UBICACION_ESCUELA <= 2; criterion = 1, statistic = 23.873
##         156) CANT_PERSONAS_HOGAR <= 4; criterion = 0.937, statistic = 18.886
##           157) VIVE_CON_PADRE == {TRUE}; criterion = 0.923, statistic = 21.257
##             158) PERCAPITA <= 371912.4; criterion = 0.985, statistic = 18.895
##               159) TRANSPORTE_ESCUELA == {escolar, público, pie, Caballo, Otro, NA}; criterion = 0.86, statistic = 20.261
##                 160)*  weights = 236 
##               159) TRANSPORTE_ESCUELA == {Carro, Bicicleta, canoa}
##                 161)*  weights = 33 
##             158) PERCAPITA > 371912.4
##               162)*  weights = 131 
##           157) VIVE_CON_PADRE == {FALSE, Muerto}
##             163)*  weights = 276 
##         156) CANT_PERSONAS_HOGAR > 4
##           164) PERCAPITA <= 330083.3; criterion = 0.89, statistic = 10.119
##             165) VIVE_CON_PADRE == {TRUE, Muerto}; criterion = 0.823, statistic = 8.828
##               166)*  weights = 289 
##             165) VIVE_CON_PADRE == {FALSE}
##               167) TRANSPORTE_ESCUELA == {escolar, público, pie}; criterion = 0.735, statistic = 13.503
##                 168)*  weights = 95 
##               167) TRANSPORTE_ESCUELA == {Carro, canoa, NA}
##                 169) EDUCACION_PADRES <= 0; criterion = 0.716, statistic = 5.967
##                   170)*  weights = 12 
##                 169) EDUCACION_PADRES > 0
##                   171)*  weights = 14 
##           164) PERCAPITA > 330083.3
##             172)*  weights = 120 
##       155) UBICACION_ESCUELA > 2
##         173) ESCUELA_OFICIAL == {SinSubstito}; criterion = 1, statistic = 22.555
##           174) CANT_PERSONAS_HOGAR <= 4; criterion = 0.986, statistic = 11.073
##             175) PERCAPITA <= 88888.89; criterion = 0.946, statistic = 8.552
##               176)*  weights = 16 
##             175) PERCAPITA > 88888.89
##               177) PERCAPITA <= 197222.2; criterion = 0.725, statistic = 5.308
##                 178)*  weights = 7 
##               177) PERCAPITA > 197222.2
##                 179) PERCAPITA <= 392500; criterion = 0.922, statistic = 7.733
##                   180)*  weights = 17 
##                 179) PERCAPITA > 392500
##                   181)*  weights = 13 
##           174) CANT_PERSONAS_HOGAR > 4
##             182)*  weights = 51 
##         173) ESCUELA_OFICIAL == {Oficial, conSubstito}
##           183) ENFERMEDAD_CRONICA == {FALSE}; criterion = 1, statistic = 18.16
##             184) CANT_PERSONAS_HOGAR <= 4; criterion = 0.995, statistic = 16.772
##               185) INCRESOS_AUTOPERCIBIDOS_HOGAR <= 2; criterion = 0.793, statistic = 9.39
##                 186)*  weights = 971 
##               185) INCRESOS_AUTOPERCIBIDOS_HOGAR > 2
##                 187) EDAD <= 8; criterion = 0.724, statistic = 5.299
##                   188)*  weights = 11 
##                 187) EDAD > 8
##                   189)*  weights = 11 
##             184) CANT_PERSONAS_HOGAR > 4
##               190) ETNIA == {5, 6}; criterion = 0.918, statistic = 14.493
##                 191) ACTIVIDAD_ULT_SEMANA == {Estudiando, Oficios_hogar, NA}; criterion = 0.713, statistic = 9.742
##                   192) ACTIVIDAD_ULT_SEMANA == {Oficios_hogar, NA}; criterion = 0.729, statistic = 8.42
##                     193)*  weights = 75 
##                   192) ACTIVIDAD_ULT_SEMANA == {Estudiando}
##                     194)*  weights = 504 
##                 191) ACTIVIDAD_ULT_SEMANA == {Otra}
##                   195)*  weights = 13 
##               190) ETNIA == {1}
##                 196) TRANSPORTE_ESCUELA == {pie, Bicicleta, NA}; criterion = 0.896, statistic = 17.773
##                   197)*  weights = 200 
##                 196) TRANSPORTE_ESCUELA == {Carro, escolar, canoa, Otro}
##                   198)*  weights = 34 
##           183) ENFERMEDAD_CRONICA == {TRUE}
##             199)*  weights = 39
\end{verbatim}

\begin{Shaded}
\begin{Highlighting}[]
\NormalTok{test\_pred }\OtherTok{\textless{}{-}} \FunctionTok{predict}\NormalTok{(modelo, }\AttributeTok{newdata =}\NormalTok{ test)}
\FunctionTok{colnames}\NormalTok{(test\_pred) }\OtherTok{\textless{}{-}} \FunctionTok{c}\NormalTok{(}\StringTok{\textquotesingle{}pred\_Satisfaccion\textquotesingle{}}\NormalTok{)}
\NormalTok{test\_pred }\OtherTok{\textless{}{-}} \FunctionTok{cbind}\NormalTok{(test\_pred, test[}\StringTok{\textquotesingle{}SATISFACCION\textquotesingle{}}\NormalTok{])}

\NormalTok{cat\_test\_pred }\OtherTok{=} \FunctionTok{lapply}\NormalTok{(test\_pred}\SpecialCharTok{/}\FloatTok{5.6}\NormalTok{, as.integer)}
\FunctionTok{table}\NormalTok{(cat\_test\_pred)}
\end{Highlighting}
\end{Shaded}

\begin{verbatim}
##                  SATISFACCION
## pred_Satisfaccion    0    1    2    3    4    5    6    7    8    9
##                 3    0    3    8   17   15    6    4    3    0    0
##                 4    1   21   62  534 1595  587  501  160   53    1
##                 5    0    9   25  434 1489  883  791  367  150   15
##                 6    0    0    1    3   16    8   24   20   14    0
\end{verbatim}

\begin{Shaded}
\begin{Highlighting}[]
\NormalTok{train\_pred }\OtherTok{\textless{}{-}} \FunctionTok{predict}\NormalTok{(modelo, }\AttributeTok{newdata =}\NormalTok{ train)}
\FunctionTok{colnames}\NormalTok{(train\_pred) }\OtherTok{\textless{}{-}} \FunctionTok{c}\NormalTok{(}\StringTok{\textquotesingle{}pred\_Satisfaccion\textquotesingle{}}\NormalTok{)}
\NormalTok{train\_pred }\OtherTok{\textless{}{-}} \FunctionTok{cbind}\NormalTok{(train\_pred, train[}\StringTok{\textquotesingle{}SATISFACCION\textquotesingle{}}\NormalTok{])}

\NormalTok{cat\_train\_pred }\OtherTok{=} \FunctionTok{lapply}\NormalTok{(train\_pred}\SpecialCharTok{/}\FloatTok{5.6}\NormalTok{, as.integer)}
\FunctionTok{table}\NormalTok{(cat\_train\_pred)}
\end{Highlighting}
\end{Shaded}

\begin{verbatim}
##                  SATISFACCION
## pred_Satisfaccion    0    1    2    3    4    5    6    7    8    9
##                 3    2   21   33   65   41   14    8    2    0    0
##                 4    2   48  132 1518 5195 1788 1369  400  112    4
##                 5    1   21   70 1225 4413 2655 2463 1102  475   26
##                 6    0    0    1   15   38   28   64   61   49    5
##                 7    0    0    0    0    1    1    2    4    4    1
\end{verbatim}

\hypertarget{cuxe1lculo-del-mse}{%
\subsubsection{Cálculo del MSE}\label{cuxe1lculo-del-mse}}

Entrenamiento

\begin{Shaded}
\begin{Highlighting}[]
\FunctionTok{print}\NormalTok{(}\StringTok{\textquotesingle{}MSE:\textquotesingle{}}\NormalTok{ )}
\end{Highlighting}
\end{Shaded}

\begin{verbatim}
## [1] "MSE:"
\end{verbatim}

\begin{Shaded}
\begin{Highlighting}[]
\NormalTok{MSE\_train }\OtherTok{\textless{}{-}} \FunctionTok{mean}\NormalTok{((train\_pred}\SpecialCharTok{$}\NormalTok{SATISFACCION }\SpecialCharTok{{-}}\NormalTok{ train\_pred}\SpecialCharTok{$}\NormalTok{pred\_Satisfaccion)}\SpecialCharTok{\^{}}\DecValTok{2}\NormalTok{)}
\FunctionTok{print}\NormalTok{(MSE\_train)}
\end{Highlighting}
\end{Shaded}

\begin{verbatim}
## [1] 42.94988
\end{verbatim}

\begin{Shaded}
\begin{Highlighting}[]
\FunctionTok{print}\NormalTok{(}\StringTok{\textquotesingle{}MS:\textquotesingle{}}\NormalTok{ )}
\end{Highlighting}
\end{Shaded}

\begin{verbatim}
## [1] "MS:"
\end{verbatim}

\begin{Shaded}
\begin{Highlighting}[]
\NormalTok{MS\_train }\OtherTok{\textless{}{-}} \FunctionTok{mean}\NormalTok{(train\_pred}\SpecialCharTok{$}\NormalTok{SATISFACCION)}\SpecialCharTok{\^{}}\DecValTok{2}
\FunctionTok{print}\NormalTok{(MS\_train)}
\end{Highlighting}
\end{Shaded}

\begin{verbatim}
## [1] 811.773
\end{verbatim}

Prueba

\begin{Shaded}
\begin{Highlighting}[]
\FunctionTok{print}\NormalTok{(}\StringTok{\textquotesingle{}MSE test:\textquotesingle{}}\NormalTok{ )}
\end{Highlighting}
\end{Shaded}

\begin{verbatim}
## [1] "MSE test:"
\end{verbatim}

\begin{Shaded}
\begin{Highlighting}[]
\NormalTok{MSE\_test }\OtherTok{\textless{}{-}} \FunctionTok{mean}\NormalTok{((test\_pred}\SpecialCharTok{$}\NormalTok{SATISFACCION }\SpecialCharTok{{-}}\NormalTok{ test\_pred}\SpecialCharTok{$}\NormalTok{pred\_Satisfaccion)}\SpecialCharTok{\^{}}\DecValTok{2}\NormalTok{)}
\FunctionTok{print}\NormalTok{(MSE\_test)}
\end{Highlighting}
\end{Shaded}

\begin{verbatim}
## [1] 45.71151
\end{verbatim}

Para las 10 categorías, entrenamiento

\begin{Shaded}
\begin{Highlighting}[]
\FunctionTok{print}\NormalTok{(}\StringTok{\textquotesingle{}MSE:\textquotesingle{}}\NormalTok{ )}
\end{Highlighting}
\end{Shaded}

\begin{verbatim}
## [1] "MSE:"
\end{verbatim}

\begin{Shaded}
\begin{Highlighting}[]
\NormalTok{MSE\_cat\_train }\OtherTok{\textless{}{-}} \FunctionTok{mean}\NormalTok{((cat\_train\_pred}\SpecialCharTok{$}\NormalTok{SATISFACCION }\SpecialCharTok{{-}}\NormalTok{ cat\_train\_pred}\SpecialCharTok{$}\NormalTok{pred\_Satisfaccion)}\SpecialCharTok{\^{}}\DecValTok{2}\NormalTok{)}
\FunctionTok{print}\NormalTok{(MSE\_cat\_train)}
\end{Highlighting}
\end{Shaded}

\begin{verbatim}
## [1] 1.623749
\end{verbatim}

\begin{Shaded}
\begin{Highlighting}[]
\FunctionTok{print}\NormalTok{(}\StringTok{\textquotesingle{}MS:\textquotesingle{}}\NormalTok{ )}
\end{Highlighting}
\end{Shaded}

\begin{verbatim}
## [1] "MS:"
\end{verbatim}

\begin{Shaded}
\begin{Highlighting}[]
\NormalTok{MS\_cat\_train }\OtherTok{\textless{}{-}} \FunctionTok{mean}\NormalTok{(cat\_train\_pred}\SpecialCharTok{$}\NormalTok{SATISFACCION)}\SpecialCharTok{\^{}}\DecValTok{2}
\FunctionTok{print}\NormalTok{(MS\_cat\_train)}
\end{Highlighting}
\end{Shaded}

\begin{verbatim}
## [1] 21.97992
\end{verbatim}

Para las 10 categorías, prueba

\begin{Shaded}
\begin{Highlighting}[]
\FunctionTok{print}\NormalTok{(}\StringTok{\textquotesingle{}MSE test:\textquotesingle{}}\NormalTok{ )}
\end{Highlighting}
\end{Shaded}

\begin{verbatim}
## [1] "MSE test:"
\end{verbatim}

\begin{Shaded}
\begin{Highlighting}[]
\NormalTok{MSE\_cat\_test }\OtherTok{\textless{}{-}} \FunctionTok{mean}\NormalTok{((cat\_test\_pred}\SpecialCharTok{$}\NormalTok{SATISFACCION }\SpecialCharTok{{-}}\NormalTok{ cat\_test\_pred}\SpecialCharTok{$}\NormalTok{pred\_Satisfaccion)}\SpecialCharTok{\^{}}\DecValTok{2}\NormalTok{)}
\FunctionTok{print}\NormalTok{(MSE\_cat\_test)}
\end{Highlighting}
\end{Shaded}

\begin{verbatim}
## [1] 1.747826
\end{verbatim}

\hypertarget{cuxe1lculo-del-mae}{%
\subsubsection{Cálculo del MAE}\label{cuxe1lculo-del-mae}}

\begin{Shaded}
\begin{Highlighting}[]
\FunctionTok{print}\NormalTok{(}\FunctionTok{mean}\NormalTok{(}\FunctionTok{abs}\NormalTok{(train\_pred}\SpecialCharTok{$}\NormalTok{SATISFACCION }\SpecialCharTok{{-}}\NormalTok{ train\_pred}\SpecialCharTok{$}\NormalTok{pred\_Satisfaccion)))}
\end{Highlighting}
\end{Shaded}

\begin{verbatim}
## [1] 5.259632
\end{verbatim}

\begin{Shaded}
\begin{Highlighting}[]
\FunctionTok{print}\NormalTok{(}\StringTok{\textquotesingle{}mean:\textquotesingle{}}\NormalTok{ )}
\end{Highlighting}
\end{Shaded}

\begin{verbatim}
## [1] "mean:"
\end{verbatim}

\begin{Shaded}
\begin{Highlighting}[]
\FunctionTok{print}\NormalTok{(}\FunctionTok{mean}\NormalTok{(train\_pred}\SpecialCharTok{$}\NormalTok{SATISFACCION))}
\end{Highlighting}
\end{Shaded}

\begin{verbatim}
## [1] 28.49163
\end{verbatim}

\begin{Shaded}
\begin{Highlighting}[]
\FunctionTok{print}\NormalTok{(}\StringTok{\textquotesingle{}MAE test:\textquotesingle{}}\NormalTok{ )}
\end{Highlighting}
\end{Shaded}

\begin{verbatim}
## [1] "MAE test:"
\end{verbatim}

\begin{Shaded}
\begin{Highlighting}[]
\FunctionTok{print}\NormalTok{(}\FunctionTok{mean}\NormalTok{(}\FunctionTok{abs}\NormalTok{(test\_pred}\SpecialCharTok{$}\NormalTok{SATISFACCION }\SpecialCharTok{{-}}\NormalTok{ test\_pred}\SpecialCharTok{$}\NormalTok{pred\_Satisfaccion)))}
\end{Highlighting}
\end{Shaded}

\begin{verbatim}
## [1] 5.433433
\end{verbatim}

\begin{Shaded}
\begin{Highlighting}[]
\FunctionTok{print}\NormalTok{(}\StringTok{\textquotesingle{}Para las 10 categorias: MAE:\textquotesingle{}}\NormalTok{ )}
\end{Highlighting}
\end{Shaded}

\begin{verbatim}
## [1] "Para las 10 categorias: MAE:"
\end{verbatim}

\begin{Shaded}
\begin{Highlighting}[]
\FunctionTok{print}\NormalTok{(}\FunctionTok{mean}\NormalTok{(}\FunctionTok{abs}\NormalTok{(cat\_train\_pred}\SpecialCharTok{$}\NormalTok{SATISFACCION }\SpecialCharTok{{-}}\NormalTok{ cat\_train\_pred}\SpecialCharTok{$}\NormalTok{pred\_Satisfaccion)))}
\end{Highlighting}
\end{Shaded}

\begin{verbatim}
## [1] 0.9373483
\end{verbatim}

\begin{Shaded}
\begin{Highlighting}[]
\FunctionTok{print}\NormalTok{(}\StringTok{\textquotesingle{}mean:\textquotesingle{}}\NormalTok{ )}
\end{Highlighting}
\end{Shaded}

\begin{verbatim}
## [1] "mean:"
\end{verbatim}

\begin{Shaded}
\begin{Highlighting}[]
\FunctionTok{print}\NormalTok{(}\FunctionTok{mean}\NormalTok{(cat\_train\_pred}\SpecialCharTok{$}\NormalTok{SATISFACCION))}
\end{Highlighting}
\end{Shaded}

\begin{verbatim}
## [1] 4.688275
\end{verbatim}

\begin{Shaded}
\begin{Highlighting}[]
\FunctionTok{print}\NormalTok{(}\StringTok{\textquotesingle{}MSE test:\textquotesingle{}}\NormalTok{ )}
\end{Highlighting}
\end{Shaded}

\begin{verbatim}
## [1] "MSE test:"
\end{verbatim}

\begin{Shaded}
\begin{Highlighting}[]
\FunctionTok{print}\NormalTok{(}\FunctionTok{mean}\NormalTok{(}\FunctionTok{abs}\NormalTok{(cat\_test\_pred}\SpecialCharTok{$}\NormalTok{SATISFACCION }\SpecialCharTok{{-}}\NormalTok{ cat\_test\_pred}\SpecialCharTok{$}\NormalTok{pred\_Satisfaccion)))}
\end{Highlighting}
\end{Shaded}

\begin{verbatim}
## [1] 0.9820972
\end{verbatim}

Cálculo del RMSE

\begin{Shaded}
\begin{Highlighting}[]
\FunctionTok{print}\NormalTok{(}\StringTok{\textquotesingle{}Para las 10 categorias: RMSE:\textquotesingle{}}\NormalTok{ )}
\end{Highlighting}
\end{Shaded}

\begin{verbatim}
## [1] "Para las 10 categorias: RMSE:"
\end{verbatim}

\begin{Shaded}
\begin{Highlighting}[]
\FunctionTok{print}\NormalTok{(}\FunctionTok{sqrt}\NormalTok{(}\FunctionTok{mean}\NormalTok{((cat\_train\_pred}\SpecialCharTok{$}\NormalTok{SATISFACCION }\SpecialCharTok{{-}}\NormalTok{ cat\_train\_pred}\SpecialCharTok{$}\NormalTok{pred\_Satisfaccion)}\SpecialCharTok{\^{}}\DecValTok{2}\NormalTok{)))}
\end{Highlighting}
\end{Shaded}

\begin{verbatim}
## [1] 1.274264
\end{verbatim}

\begin{Shaded}
\begin{Highlighting}[]
\FunctionTok{print}\NormalTok{(}\StringTok{\textquotesingle{}RMSE test:\textquotesingle{}}\NormalTok{ )}
\end{Highlighting}
\end{Shaded}

\begin{verbatim}
## [1] "RMSE test:"
\end{verbatim}

\begin{Shaded}
\begin{Highlighting}[]
\FunctionTok{print}\NormalTok{(}\FunctionTok{sqrt}\NormalTok{(}\FunctionTok{mean}\NormalTok{((cat\_test\_pred}\SpecialCharTok{$}\NormalTok{SATISFACCION }\SpecialCharTok{{-}}\NormalTok{ cat\_test\_pred}\SpecialCharTok{$}\NormalTok{pred\_Satisfaccion)}\SpecialCharTok{\^{}}\DecValTok{2}\NormalTok{)))}
\end{Highlighting}
\end{Shaded}

\begin{verbatim}
## [1] 1.322054
\end{verbatim}

\hypertarget{cuxe1lculo-de-la-matriz-de-confusiuxf3n-de-los-datos}{%
\subsubsection{Cálculo de la Matriz de confusión de los
datos}\label{cuxe1lculo-de-la-matriz-de-confusiuxf3n-de-los-datos}}

Entrenamiento

\begin{Shaded}
\begin{Highlighting}[]
\CommentTok{\# to factor}
\NormalTok{cat\_train\_pred}\SpecialCharTok{$}\NormalTok{SATISFACCION }\OtherTok{=} \FunctionTok{factor}\NormalTok{(cat\_train\_pred}\SpecialCharTok{$}\NormalTok{SATISFACCION)}
\NormalTok{cat\_train\_pred}\SpecialCharTok{$}\NormalTok{pred\_Satisfaccion }\OtherTok{=} \FunctionTok{factor}\NormalTok{(cat\_train\_pred}\SpecialCharTok{$}\NormalTok{pred\_Satisfaccion)}

\CommentTok{\#reorder}
\NormalTok{cat\_train\_pred}\SpecialCharTok{$}\NormalTok{pred\_Satisfaccion }\OtherTok{=} \FunctionTok{factor}\NormalTok{(cat\_train\_pred}\SpecialCharTok{$}\NormalTok{pred\_Satisfaccion, }
                                          \AttributeTok{levels=}\FunctionTok{c}\NormalTok{(}\DecValTok{0}\NormalTok{,}\DecValTok{1}\NormalTok{,}\DecValTok{2}\NormalTok{,}\DecValTok{3}\NormalTok{,}\DecValTok{4}\NormalTok{,}\DecValTok{5}\NormalTok{,}\DecValTok{6}\NormalTok{,}\DecValTok{7}\NormalTok{,}\DecValTok{8}\NormalTok{,}\DecValTok{9}\NormalTok{))}

\CommentTok{\#create confusion matrix}
\FunctionTok{confusionMatrix}\NormalTok{(}\FunctionTok{table}\NormalTok{(cat\_train\_pred))}
\end{Highlighting}
\end{Shaded}

\begin{verbatim}
## Confusion Matrix and Statistics
## 
##                  SATISFACCION
## pred_Satisfaccion    0    1    2    3    4    5    6    7    8    9
##                 0    0    0    0    0    0    0    0    0    0    0
##                 1    0    0    0    0    0    0    0    0    0    0
##                 2    0    0    0    0    0    0    0    0    0    0
##                 3    2   21   33   65   41   14    8    2    0    0
##                 4    2   48  132 1518 5195 1788 1369  400  112    4
##                 5    1   21   70 1225 4413 2655 2463 1102  475   26
##                 6    0    0    1   15   38   28   64   61   49    5
##                 7    0    0    0    0    1    1    2    4    4    1
##                 8    0    0    0    0    0    0    0    0    0    0
##                 9    0    0    0    0    0    0    0    0    0    0
## 
## Overall Statistics
##                                           
##                Accuracy : 0.34            
##                  95% CI : (0.3339, 0.3461)
##     No Information Rate : 0.4126          
##     P-Value [Acc > NIR] : 1               
##                                           
##                   Kappa : 0.0706          
##                                           
##  Mcnemar's Test P-Value : NA              
## 
## Statistics by Class:
## 
##                      Class: 0 Class: 1 Class: 2 Class: 3 Class: 4 Class: 5
## Sensitivity          0.000000 0.000000  0.00000 0.023025   0.5362   0.5918
## Specificity          1.000000 1.000000  1.00000 0.994142   0.6104   0.4842
## Pos Pred Value            NaN      NaN      NaN 0.349462   0.4916   0.2132
## Neg Pred Value       0.999787 0.996167  0.98995 0.881595   0.6520   0.8340
## Prevalence           0.000213 0.003833  0.01005 0.120235   0.4126   0.1911
## Detection Rate       0.000000 0.000000  0.00000 0.002768   0.2213   0.1131
## Detection Prevalence 0.000000 0.000000  0.00000 0.007922   0.4501   0.5303
## Balanced Accuracy    0.500000 0.500000  0.50000 0.508584   0.5733   0.5380
##                      Class: 6  Class: 7 Class: 8 Class: 9
## Sensitivity          0.016385 0.0025494  0.00000 0.000000
## Specificity          0.989935 0.9995892  1.00000 1.000000
## Pos Pred Value       0.245211 0.3076923      NaN      NaN
## Neg Pred Value       0.834525 0.9333078  0.97274 0.998467
## Prevalence           0.166361 0.0668257  0.02726 0.001533
## Detection Rate       0.002726 0.0001704  0.00000 0.000000
## Detection Prevalence 0.011116 0.0005537  0.00000 0.000000
## Balanced Accuracy    0.503160 0.5010693  0.50000 0.500000
\end{verbatim}

Prueba

\begin{Shaded}
\begin{Highlighting}[]
\CommentTok{\# to factor}
\NormalTok{cat\_test\_pred}\SpecialCharTok{$}\NormalTok{SATISFACCION }\OtherTok{=} \FunctionTok{factor}\NormalTok{(cat\_test\_pred}\SpecialCharTok{$}\NormalTok{SATISFACCION)}
\NormalTok{cat\_test\_pred}\SpecialCharTok{$}\NormalTok{pred\_Satisfaccion }\OtherTok{=} \FunctionTok{factor}\NormalTok{(cat\_test\_pred}\SpecialCharTok{$}\NormalTok{pred\_Satisfaccion)}

\CommentTok{\#reorder}
\NormalTok{cat\_test\_pred}\SpecialCharTok{$}\NormalTok{SATISFACCION }\OtherTok{=} \FunctionTok{factor}\NormalTok{(cat\_test\_pred}\SpecialCharTok{$}\NormalTok{SATISFACCION, }
                                    \AttributeTok{levels=}\FunctionTok{c}\NormalTok{(}\DecValTok{0}\NormalTok{,}\DecValTok{1}\NormalTok{,}\DecValTok{2}\NormalTok{,}\DecValTok{3}\NormalTok{,}\DecValTok{4}\NormalTok{,}\DecValTok{5}\NormalTok{,}\DecValTok{6}\NormalTok{,}\DecValTok{7}\NormalTok{,}\DecValTok{8}\NormalTok{,}\DecValTok{9}\NormalTok{))}

\NormalTok{cat\_test\_pred}\SpecialCharTok{$}\NormalTok{pred\_Satisfaccion }\OtherTok{=} \FunctionTok{factor}\NormalTok{(cat\_test\_pred}\SpecialCharTok{$}\NormalTok{pred\_Satisfaccion, }
                                         \AttributeTok{levels=}\FunctionTok{c}\NormalTok{(}\DecValTok{0}\NormalTok{,}\DecValTok{1}\NormalTok{,}\DecValTok{2}\NormalTok{,}\DecValTok{3}\NormalTok{,}\DecValTok{4}\NormalTok{,}\DecValTok{5}\NormalTok{,}\DecValTok{6}\NormalTok{,}\DecValTok{7}\NormalTok{,}\DecValTok{8}\NormalTok{,}\DecValTok{9}\NormalTok{))}

\CommentTok{\#create confusion matrix}
\FunctionTok{confusionMatrix}\NormalTok{(}\FunctionTok{table}\NormalTok{(cat\_test\_pred))}
\end{Highlighting}
\end{Shaded}

\begin{verbatim}
## Confusion Matrix and Statistics
## 
##                  SATISFACCION
## pred_Satisfaccion    0    1    2    3    4    5    6    7    8    9
##                 0    0    0    0    0    0    0    0    0    0    0
##                 1    0    0    0    0    0    0    0    0    0    0
##                 2    0    0    0    0    0    0    0    0    0    0
##                 3    0    3    8   17   15    6    4    3    0    0
##                 4    1   21   62  534 1595  587  501  160   53    1
##                 5    0    9   25  434 1489  883  791  367  150   15
##                 6    0    0    1    3   16    8   24   20   14    0
##                 7    0    0    0    0    0    0    0    0    0    0
##                 8    0    0    0    0    0    0    0    0    0    0
##                 9    0    0    0    0    0    0    0    0    0    0
## 
## Overall Statistics
##                                           
##                Accuracy : 0.3221          
##                  95% CI : (0.3118, 0.3326)
##     No Information Rate : 0.3983          
##     P-Value [Acc > NIR] : 1               
##                                           
##                   Kappa : 0.0548          
##                                           
##  Mcnemar's Test P-Value : NA              
## 
## Statistics by Class:
## 
##                       Class: 0 Class: 1 Class: 2 Class: 3 Class: 4 Class: 5
## Sensitivity          0.0000000  0.00000  0.00000 0.017206   0.5120   0.5950
## Specificity          1.0000000  1.00000  1.00000 0.994292   0.5919   0.4823
## Pos Pred Value             NaN      NaN      NaN 0.303571   0.4538   0.2121
## Neg Pred Value       0.9998721  0.99578  0.98772 0.874936   0.6469   0.8357
## Prevalence           0.0001279  0.00422  0.01228 0.126343   0.3983   0.1898
## Detection Rate       0.0000000  0.00000  0.00000 0.002174   0.2040   0.1129
## Detection Prevalence 0.0000000  0.00000  0.00000 0.007161   0.4495   0.5324
## Balanced Accuracy    0.5000000  0.50000  0.50000 0.505749   0.5520   0.5387
##                      Class: 6 Class: 7 Class: 8 Class: 9
## Sensitivity          0.018182  0.00000  0.00000 0.000000
## Specificity          0.990462  1.00000  1.00000 1.000000
## Pos Pred Value       0.279070      NaN      NaN      NaN
## Neg Pred Value       0.832428  0.92967  0.97225 0.997954
## Prevalence           0.168798  0.07033  0.02775 0.002046
## Detection Rate       0.003069  0.00000  0.00000 0.000000
## Detection Prevalence 0.010997  0.00000  0.00000 0.000000
## Balanced Accuracy    0.504322  0.50000  0.50000 0.500000
\end{verbatim}

\begin{Shaded}
\begin{Highlighting}[]
\FunctionTok{levels}\NormalTok{(cat\_test\_pred}\SpecialCharTok{$}\NormalTok{SATISFACCION)}
\end{Highlighting}
\end{Shaded}

\begin{verbatim}
##  [1] "0" "1" "2" "3" "4" "5" "6" "7" "8" "9"
\end{verbatim}

\hypertarget{conclusiones}{%
\section{Conclusiones}\label{conclusiones}}

\hypertarget{recomendaciones}{%
\section{Recomendaciones}\label{recomendaciones}}

\begin{thebibliography}{X}
\bibitem{b1}
Ramírez Pérez, Mauricio; Lee Maturana, Sau-Lyn (2012). Factores asociados a la satisfacción vital en adultos mayores de 60 años. Polis (Santiago), 11(33), 407–428. doi:10.4067/s0718-65682012000300020
\bibitem{b2}
Kutubaeva RZh (2019) Analysis of life satisfaction of the elderly population on the example of Sweden, Austria and Germany. Population and Economics 3(3): 102-116. https://doi.org/10.3897/popecon.3.e47192
\bibitem{b3}
Palmore, E., Luikart, C. (1972). Health and Social Factors Related to Life Satisfaction. Journal of Health and Social Behavior, 13(1), 68–80. doi: 10.2307/2136974
\bibitem{b4}
Naidu, Aditi (2009). Factors affecting patient satisfaction and healthcare quality. International Journal of Health Care Quality Assurance, 22(4), 366–381. doi:10.1108/09526860910964834 
\bibitem{b5}
ROBLES-GARCIA, Monica et al. Variables relacionadas con la satisfaccion laboral: un estudio transversal a partir del modelo EFQM. Gac Sanit [online]. 2005, vol.19, n.2, pp.127-134. ISSN 0213-9111
\bibitem{b6}
Booth, Jaime; Ayers, Stephanie L.; and Marsiglia, Flavio F. (2012) "Perceived Neighborhood Safety and Psychological Distress: Exploring Protective Factors," The Journal of Sociology \& Social Welfare: Vol. 39 : Iss. 4 , Article 8. Available at: https://scholarworks.wmich.edu/jssw/vol39/iss4/
\bibitem{b7}
https://ourworldindata.org/happiness-and-life-satisfaction
\end{thebibliography}

\end{document}
